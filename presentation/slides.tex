% titlepage-demo.tex
\documentclass{beamer}
\usetheme{Montpellier}
% \usetheme{Copenhagen}
\usepackage{helvet}
%\usepackage[x11names, rgb]{xcolor}
\usepackage[utf8]{inputenc}
\usepackage{tikz}
\usepackage{graphicx}
\usepackage{tipa}
\usetikzlibrary{snakes,arrows,shapes}
\usepackage{amsmath}
\usepackage[absolute,overlay]{textpos} 
\newenvironment{reference}[2]{% 
  \begin{textblock*}{\textwidth}(#1,#2) 
      \footnotesize\it\bgroup\color{red!50!black}}{\egroup\end{textblock*}} 
% items enclosed in square brackets are optional; explanation below
\title[OCaml, a not-\textit{too}-informed opinion]{
OCaml, a not-\textit{too}-informed opinion
}% par none, Señor}
\author[Arora]{\underline{Div}yanshu Arora}
\institute[NUS]{
%  $3\frac{1}{2}$ year, Computer Science\\
  \texttt{arora@comp.nus.edu.sg}
}
% \date[October 2012]{October 16, 2012}


\AtBeginSection[]
{
  \begin{frame}<beamer>
    \frametitle{Outline for section \thesection}
    \tableofcontents[currentsection,currentsubsection]
  \end{frame}
}


\begin{document}

%--- the titlepage frame -------------------------%
\begin{frame}[plain]
  \titlepage
\end{frame}

%--- the presentation begins here ----------------%
% \setbeamertemplate{footline}{\vspace*{-1cm}\centering Some information\\ some more info\par} 
\begin{frame}<beamer>
  \frametitle{Outline of topics covered}
  \tableofcontents
\end{frame}

\section{The Pros}
\label{sec:pro}

\begin{frame}{}
  \begin{itemize}
  \item Type inference
    \pause
  \item Type safety checks \textit{(level: awesome)}
    \pause
    \begin{itemize}
    \item Even more awesome with sum types
    \end{itemize}
    \pause
  \item Really good performance \& memory footprint, especially for
    its feature set. \textit{(with caveats...)}
    \pause
  \item All this results in terse, readable code
    \pause
  \item camlp4 :D
  \end{itemize}
\end{frame}

\section{The downside}
\label{sec:downside}

\begin{frame}{Build Tools}
  \begin{itemize}
  \item Makefiles, via \texttt{gmake} (what we've been using)
  \item Abstraction abstraction abstraction! \texttt{autoconf}/\texttt{automake}
    \pause
  \item But wait! There's also \texttt{omake}, which is sort of like make with
    batteries included, especially for ocaml projects.
    \pause
  \item There's also \texttt{oasis}, to ``integrate a configure, build and
    install system in your OCaml project.''
    \pause
  \item \texttt{ocamlbuild}, ``automating the compilation of most OCaml
    projects with minimal user input.''
    \pause
  \item Using flaps of a butterfly's wings to flip magnetic disk bits.
    \pause
  \end{itemize}
  Compared to cabal/Hackage, a tad lacking.
\end{frame}

\begin{frame}{Performance--the caveats}
  \begin{itemize}
  \item GIL galore!
    \pause
  \item Multi-core support isn't on the roadmap. Preferred manner is
    message-passing, between multiple processes (via
    \texttt{Functory}, \texttt{OCamlnet})
    \pause
  \item But there's also \texttt{Netmulticore}, which uses mapped
    shared memory, and \texttt{OCaml-Java}
  \end{itemize}
\end{frame}

\begin{frame}{}
  \begin{block}{}
    This might not be a downside at all!\\
    Perl, with its whale guts, is doing well enough.
  \end{block}
  \begin{block}{Generic Pointers, or the lack thereof}
    a la \texttt{string\_of\_dPL}, \texttt{string\_of\_dPL\_type\_decl}
  \end{block}
\end{frame}

\end{document}
